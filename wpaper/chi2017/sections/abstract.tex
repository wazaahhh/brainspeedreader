\begin{abstract}
When exposed large amounts of information from heterogenous sources, many people aim to process more in less time. Multi-tasking is one way to do more, but it has been found to be detrimental to concentration. We alleviate this short-term versus long-term reward tradeoff with the help of a brain-computer interface (BCI), inspired by neurofeedback techniques. This BCI allows for speed reading while training concentration in a seamless way: Users of the {\it brain speed reader} (BSR) control the speed of a fast-paced sequence of coherent stimuli, presented on a screen using rapid serial visual presentation (RSVP). We test two different neurofeedback mechanisms. Even without preliminary BCI training, a majority of users effectively control the pace of stimuli presentation. We find that achieving self-regulation is negatively associated with age, text length and, on the contrary, positively associated with topic familiarity and reading comfort. Our results help better delineate demographics and potential future uses of this novel brain computer interface.
\end{abstract}

%- time to read normal text
%- comprehension questions
%- ratings & ranks
%- demographics
%- achieve stability
%- rate | achieve stability
%- time to achieve stability (implied calibration)
%- achieve stability and then lose it ?
