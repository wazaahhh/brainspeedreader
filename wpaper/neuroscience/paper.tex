% Template for PLoS
% Version 1.0 January 2009
%
% To compile to pdf, run:
% latex plos.template
% bibtex plos.template
% latex plos.template
% latex plos.template
% dvipdf plos.template

\documentclass[12pt]{article}

% amsmath package, useful for mathematical formulas
\usepackage{amsmath}
% amssymb package, useful for mathematical symbols
\usepackage{amssymb}
\usepackage{url}
% graphicx package, useful for including eps and pdf graphics
% include graphics with the command \includegraphics
\usepackage{graphicx}

% cite package, to clean up citations in the main text. Do not remove.
\usepackage{cite}

\usepackage{color} 

% Use doublespacing - comment out for single spacing
%\usepackage{setspace} 
%\doublespacing


% Text layout
\topmargin 0.0cm
\oddsidemargin 0.5cm
\evensidemargin 0.5cm
\textwidth 16cm 
\textheight 21cm

% Bold the 'Figure #' in the caption and separate it with a period
% Captions will be left justified
\usepackage[labelfont=bf,labelsep=period,justification=raggedright]{caption}

% Use the PLoS provided bibtex style
\bibliographystyle{plos2009}

% Remove brackets from numbering in List of References
\makeatletter
\renewcommand{\@biblabel}[1]{\quad#1.}
\makeatother


% Leave date blank
\date{}

\pagestyle{myheadings}
%% ** EDIT HERE **


%% ** EDIT HERE **
%% PLEASE INCLUDE ALL MACROS BELOW

%% END MACROS SECTION

\begin{document}

% Title must be 150 characters or less
\begin{flushleft}
{\Large
\textbf{Neurofeedback, Rapid Serial Visual Presentation and Cognition}
}
% Insert Author names, affiliations and corresponding author email.
\\
Thomas Maillart$^{1,\ast}$, 
Nick Merrill$^{1}$, 
John Chuang$^{1}$
\\
\bf{1} School of Information, UC Berkeley, Berkeley, CA, United States
%\\
%\bf{2} Author2 Dept/Program/Center, Institution Name, City, State, Country
%\\
%\bf{3} Author3 Dept/Program/Center, Institution Name, City, State, Country
%\\
$\ast$ E-mail: Corresponding maillart@berkeley.edu
\end{flushleft}

% Please keep the abstract between 250 and 300 words
\begin{abstract}
At the age of online information abundance, the human capacity to retain knowledge is largely limited by the time and the attention required to read text, watch videos, listen to podcasts. For written information, rapid serial visual presentation (RSVP) helps greatly save time with similar levels of text understanding, compared with traditional reading. However, RSVP does not account for attention. We present a simple hybrid brain-computer interface (BCI) that controls in real-time the speed of reading by measuring the instant level of higher cognitive brain activity. Electroencephalogram (EEG) signal is acquired with a single channel consumer-grade headset and analyzed in the frequency domain. The pace of word display is controlled by a measure brainwave entropy. We have conducted a controlled experiment with 50 subjects with three distinct treatments, and we show that brain-controlled speed-reading increases the speed and the understanding of texts by subjects.
\end{abstract}
% Please keep the Author Summary between 150 and 200 words
% Use first person. PLoS ONE authors please skip this step. 
% Author Summary not valid for PLoS ONE submissions.   
%\section*{Author Summary}

\section{Introduction}
Striking the right balance between skimming through newspaper articles, blog posts, tweets, on the one hand, and focusing attention on the most important information contents on the other hand, is an increasingly though challenge in a world of limited time \cite{maillart2011} and attention \cite{anham2006economics}. One way can overcome information overflow by applying filters tailored on individual's past interests \cite{}. This algorithmic approach is however increasingly criticized for generating positive feedback loops and so-called filter bubbles, leaving people exposed to more of the same information \cite{}. Another approach consists in optimizing individual exposure to information, in a way that more knowledge can be processed for the same amount of time.

Speed reading technologies based on rapid serial visual presentation (RSVP) have been developed to increase the throughput of information delivered to people's eyes \cite{slate2014}. {\bf [say more about current speed-reading technologies]}. People choose the speed in their comfort zone (usually around 125 milliseconds per word) prior to reading the text. If the text appears to be harder than expected a slower speed would have been desirable. On the contrary, an easy or boring text, may not deserve as much time, and speed reading could go significantly faster.

Current speed-reading technologies lack the seamless speed control needed to provide online optimization of time spent on each word displayed, or at least on portions of sentences and paragraphs. Interestingly, the wish to seamlessly control RSVP dates back to the very invention of this technique by early cognitive scientists \cite{}.

Brain computer interfaces (BCI) are well-known for being seamless. They have primarily been developed to help heavily motor-impaired recover some communication capabilities, in a way that no physical input is required from the individual \cite{}. Other types of BCI involve neurofeedback, which is used as a remediation technique for people suffering mainly from the Attention Deficit Hyperactivity Disorder (ADHD) \cite{}. Most consumer-grade applications, fostering ``attention" and ``meditation" training using brainwaves rely on similar mechanisms \cite{}.

Controlling speed reading requires however to process signal almost as fast a it is delivered by EEG signal, which is of the order of hundred milliseconds. Moreover, this processing must be lightweight enough to run on a normal computer or mobile device, without slowing the presentation of words on the reader's screen. To enhance usability, the approach should avoid long and boring calibration procedures, and should be efficient from the start.

We address the problem of online optimization of speed reading, with a lightweight algorithm, which guaranties real-time adaption of the rate of word presentation as a function of cognitive activity as captured by single-channel EEG device.

This paper makes 2 primary research contributions: 

(1) It establishes the {\it feasibility} of speed reading system seamlessly controlled by single-channel EEG signal, and 

(2) it exhibits improvements in {\it understanding, comfort and speed} compared to speed reading with constant rate of word display.


%
%The core problem is the time required to actually read, process and memorize information for future restitution, and RSVP was primarily invented for the purpose of studying the fine-grained memory processes at work when people read text \cite{}, listen to audio streams \cite{}, or are presented with images \cite{}. In the case of language, in particular words displayed one after the other, it appears that the time-gains come from reduced eye movement, as the focus remains in a narrow area where the word is displayed \cite{}. Research in cognitive science shows that words can be presented at as fast as XXX words per second without significant loss of understanding and integration (see Section \ref{} for more detailed review of literature, and precisions regarding the nature of understanding and integration: recall, conceptual understanding, etc.). 
%
%Nevertheless, some texts contain words that are most difficult than others, which require extra memory and conceptual processing after each word (resp. group of words). For instance, a short pause at the end of sentences (resp. paragraphs), considerably helps understanding and recall \cite{}. In other words, the capacity to understand a text stems for an adequate optimization (minimization) of time required for integrating knowledge, which might also differ from one subject to another. This optimization can be made manually (i.e., set the number of words per second) at the level of several texts, of one text, maybe at the level of a paragraph, but hardly at the word level, since the time required to set the pace would eliminate the gains obtained from using RSVP. Also, optimization at the text or paragraph levels requires prior knowledge on the text by the user, which is unpractical since the {\it a priori} goal of RSVP is not consolidating knowledge, but rather going quickly through information.
% 
%We are therefore left with three solutions, which consist in (i) setting an average word pace for all words and all text (this average word pace can be manually set/optimized by the user, (ii) programmatically infer the time required to integrate the meaning of a word \cite{smith2013effect}, or (iii) sharply reduce the cost of optimizing the pace-of-word.
%
%The emergence of consumer-grade Brain Computer Interfaces (cBCI) open new opportunities for such seamless and fine-grained control, beyond medical or lab experimentations. Although usual BCIs rely on medical grade devices, we have previously shown the feasibility of cBCI interface with cheap consumer-grade EEG devices, relying on \textcolor{red}{\bf [complete sentence here]} \cite{}. We shall expand this method, using a continuum of entropy-based attention metrics, to compute in real-time the level of attention around each word presented and control the pace of words accordingly, in a continuous optimization process.
%
%



\section{Method \& apparatus}
\label{method}
Past research has shown that both rapid serial visual presentation (RSVP) and neurofeedback (NF) have positive effects on reading and concentration. Our approach posits that a RSVP controlled by self-regulated neurofeedback may help build a new kind of brain-computer interface (BCI), which would allow people read fast, yet at their own pace, and at the same time, maintain engagement. Thus readers can avoid multitasking, and keep concentrated on reading (e.g., a long newspaper article).

We hypothesize that meaningful brain wave modulations can be extracted from EEG recordings obtained from even the simplest consumer-grade brain scanner. This signal is then used to update the rate of RSVP in real-time. The updated RSVP rate in turns influences brain wave modulations (see Figure \ref{fig:apparatus}). The influence loop between the brain (i.e., wave modulations) and the computer (i.e., RSVP rate) variables constitute a feed-back loop, which is a typical neurofeedback feature. If the RSVP rate stabilizes, then self-regulation is achieved. On the contrary, if the rate drifts away (i.e., slower or faster rate), then a positive (resp. negative) feedback loop occurs: This is a signature of failure to self-regulate, and thus, to effectively synchronize with the task at hand. Our main aim here is precisely to determine how well users actually achieve self-regulation, and how it influences their speed reading experience, including comprehension and recall.

Our approach is novel in a number of ways, and thus, carries as many technical challenges. Most BCIs involve having a specific mental gesture (see e.g., \cite{chuang2013ithink,jonhson2014mythoughts}) or a stimulus (e.g., \cite{}) elicit a corresponding response by the computer. This approach requires some training by the user to repeat the gesture and/or for the computer to recognize the mental state associated with a stimulus. On the contrary, with the {\it brain speed reader} the ordered sequence of stimuli (i.e., words) triggers small incremental reactions and rate changes, as a result of learning {\it on the fly} and control through self-regulation.

\subsection{Avoiding ``under-control"}
The main challenge for such novel BCI lies in the capability to process information from EEG recordings at a speed in adequacy with the RSVP rate, in order to avoid ``under-control": On the contrary to over-control,\footnote{A typical example of over-control is when it takes several minutes (resp. hours) for the heating system of a house to adjust the temperature in a room after a quick thermostat adjustment by the operator. The operator may be tempted to adjust too often or too much to compensate the slow temperature change, thus leading to huge temperature amplitudes, or even resonance phenomena.} if acquiring, processing information, and taking a control decision takes more time than the evolution pace of the state variable (i.e., here, the RSVP rate), then control action is always behind and thus inefficient. Molly Potter, one the pioneers of cognitive science using RSVP, pointed out early that a control in real-time of the RSVP presentation rate would be desirable, both for science and application purposes. However, she acknowledged that any mechanical means (e.g., through keystrokes, or eye-blinking) would take to much time and introduce too much delay in comparison with the presentation rate, hence leading to a form of under-control \cite{potter1984rapid}.

\subsection{Quality, processing \& usability tradeoffs}
Even if no mechanical input is involved and signal used for RSVP rate control is sourced directly from the brain, avoiding under-control is still a challenge: EEG signal is notoriously noisy, and signal quality and resolution improves significantly with the number and the quality of electrodes \cite{michel2004eeg}. Yet, each supplementary electrode requires additional processing time, which increase at best linearly, but most likely at the square of the number of electrodes, if correlations between EEG signals from electrodes exist, which is most often the case. There is thus a trade-off between input quality and processing time, taking into account that poor signal quality may require additional data cleaning and thus, processing time. The choice of recording technology influences qualitatively which approaches can be used to analyze EEG signal: For instance, low quality EEG equipment can hardly detect event-related potentials (ERPs), even if these ERPs are repeated multiple times to average out noise \cite{ERP_consumergrade_EEG}. The design choice is also constrained by time: The average RSVP rate is 125 milliseconds (ms) per word. EEG signal must therefore take less than 125 ms to process and deliver the next rate update, and the less processing time, the more the BCI is in sync with ``orders" provided by the brain. Finally, the more sophisticated the EEG recording device used, the less obvious and wearable the form factor. Hence, in the case of EEG, sophistication comes at the cost of usability, potential scaling limitation of the user base, and thus, limitation of social benefits of the brain speed reader.

\subsection{Entropy as an efficient way to compress the EEG signal}
Here, we operate and test the brain speed reader with the cheapest consumer-grade EEG headset available on the market, with one dry electrode placed on position {Fp1} in the 10--20 electrode system \cite{klem1999ten}. Namely, we try to elicit self-regulation with the simplest and most accessible EEG recording device (Neurosky Mindwave, approximative budget: \$100). The EEG signal recorded at 512 Hz is processed in the spectral domain: The Shannon entropy of the power spectrum of the last quarter second of EEG signal (128 measures) is computed as, 

\begin{equation}
\label{eq:shannon}
S = - \sum_{i=1}^n p_i\cdot log_{2}(p_i),
\end{equation}

with $p_i$ is the activity level of the i-est frequency between $0$ and $128$ Hz (resampled and normalized following \cite{merrill2015}). $S$ is then normalized into 

\begin{equation}
\label{eq:Snormalized}
S_{norm}(t) = \frac{S(t) - \langle S \rangle}{\sigma_{S}}, 
\end{equation}

with $\langle S \rangle$ and $\sigma_{S}$ the average and standard deviation of the entropy $S$ calculated over the last 5 seconds. $S_{norm}(t)$ ensures that only deviations of brain wave modulations from the average influence the RSVP rate. The measure of entropy is particularly handy because it {\it compresses} information contained in a whole density function into a scalar. Entropy is thus one simple and non-parametric way to account for the variation of power spectrum density function, given that the density function of brain wave frequencies is roughly pink noise, which follows a power law heavy-tailed density function given by $pdf(f) \sim 1/f^{\mu+1}$, with $f$ the frequency and $\mu \approx 1$. When working with the power spectrum of brain waves, researchers tend to dismiss frequencies higher than 60Hz, because higher values are usually considered as polluted by artifacts, in particular as a result of facial muscles contractions. Researchers tend to be even more restrictive down to narrow frequency bands (see Background section). Here, we assume that high frequency muscular artifacts are idiosyncratic and may not influence the RSVP rate in a systematic way, because at the high frequencies they occur, they cannot directly be consciously controlled, and they may only influence locally in time the RSVP rate in a random way, as part of the background noise. Furthermore, at high frequencies, the contribution to entropy is rather small. High amplitude, low frequency artifacts such as eye blinks are detected by the amplitude of the signal they generate. Sections of signal with $|amplitude| > 150\mu v$ are dropped and the latest available measure of $S_{norm}$ is used. To ensure that the RSVP rate is not slowed, the computation of $S_n$ is performed in a separate thread. The most recent measure $S_n$ available is used to determine the display time of the next stimulus in the sequence of words is displayed. 


%by computing the power spectrum density associated with the Fourier transform of the signal over each period. Figure \ref{fig:pspectrum} shows typical power spectrum densities, associated with simple tasks, such as {\it resting state} (i.e. closed eyes, no mental focus) \cite{}, {\it passively watch a screen (with a video?)}, and read a text in English on the same screen, in silence and loud. We see that the structure of the power spectrum density varies significantly across tasks. This result is consistent also across subjects \textcolor{red}{[\bf bring evidence here.]}.

%\begin{figure}[!t]
%\centering
%%%\includegraphics[width=1.1\columnwidth]{../figures/compareArtifacts.eps}
%\caption{Show the difference between resting state, being passive in front of a screen, and actively reading a text, in double logarithmic scale.}
%\label{fig:pspectrum}
%\end{figure}

%Figure \ref{fig:pspectrum} shows the heavy-tailed distribution of the power spectrum density, approximately a straight line in double logarithmic scale, which corresponds to current understanding of the pink noise nature of brain synchronization, $pdf(f) \sim 1/f^{\mu+1}$, with $f$ the frequency and $\mu \approx 1$ \cite{pink noise brain}.

\subsection{RSVP rate control mechanism}
With the normalized entropy $S_{norm}$, we have the most compact measure (i.e., a scalar) of brain state changes over a short period, as a result of the normalization (\ref{eq:Snormalized}). Although it is a rough {\it compressed} measure it captures all systematic changes of brain wave modulations (perturbations are considered to occur ranfomly). We shall now see how $S_{norm}$ controls the RSVP rate. 

We define $X = X(t)$ the presentation duration for each word. At each iteration, X is updated according to,

\begin{equation}
X(t+\Delta t) = X(t) \left[1 + \alpha \cdot S_{norm}(t)\right],
\label{eq:RateChange}
\end{equation}

with $S_{norm}$, the normalized entropy, continually updated according to (\ref{eq:Snormalized}), $\alpha$ is a small fixed parameter which determines the negative (resp. positive) incremental influence of $S_{norm}$ on $X(t)$. If $\alpha < 0$ and $S_{norm} > 0$, the RSVP rate decreases. Conversely, if $\alpha > 0$ and $S_{norm} > 0$, then the RSVP rate increases. Here, $|\alpha|$ is set to 0.005. We shall determine what sign of $\alpha$ is most convenient to users. If the median value of the EEG signal is superior to $150\mu v$, the measure of entropy $S$ is dropped, and the rate $X(t)$ is not updated. Separating sentences and paragraphs with small pauses is crucial to let subjects gain better understanding of the whole text \cite{}. After each sentence (resp. paragraph), we added the equivalent time of two (resp. four) words. For example, if the current rate is 150ms/word, then a pause of 300ms is used as a break between two sentences. 

Figure \ref{fig:apparatus} summarizes the apparatus and its step-by-step functioning: (i) the EEG signal is recorded, (ii)  processed online every quarter second  to obtain a power spectrum density, which is in turn (iii) represented by the normalized entropy  $S_{norm}$; (iv) depending on the values of parameter $\alpha$ and the random variable $S_{norm}$, the {\it brain speed reader} will change the RSVP rate. $X(t=0)$ is set to $125$ ms/word, which has been determined to be the comfort zone for RSVP reading \cite{kujala2007phase}.

\begin{figure}[!h]
\centering
\includegraphics[width=0.9\columnwidth]{../figures2/apparatus.eps}
\caption{Brain speed-reader apparatus: {\bf (a)} Words are displayed and read one after the other at a given rate. {\bf (b)} the EEG signal is recorded through a consumer grade device (here the {\it Neurosky Mindwave}). {\bf (c)} The EEG signal is turned every 0.5 seconds into a power spectrum through a Fourier transform, {\bf (d)} the characteristics of the power spectrum are compressed into a single value characteristic entropy $s$ value. {\bf (e)} A new rate of word display is updated by taking into its current value and $s$. {\bf (f)} The rate of word display is updated accordingly.}
\label{fig:apparatus}
\end{figure}


\subsection{RSVP rate self-regulation}
Self-regulation neurofeedback is not granted to everyone, and we expect that some users may experience difficulty to achieve it. We therefore define a {\it stability} threshold, which measure is aimed at minimizing the ratio of the square formed by the two most extreme rates ($r_M$ for maximum rate and $r_m$ for the minimum rate), their time coordinates (resp. $k_{r_M}$ and  $k_{r_m}$) and by the number of words as the denominator. Although quite {\it ad-hoc}, this definition of stability encompasses transient rate changes, in particular those occurring when the brain speed reader starts as it takes some time for users to achieve self-regulation (see Figure \ref{fig:stability} for an illustration). By visual assessment, we consider that self-regulation is achieved for $stability < 2$ , although $stability$ is a continuous variable and the smaller its value, the better self-regulation. If the user reaches the upper or lower boundary (resp. $r_M = 30$  and $r_m = 175$ ms/word), then stability is arbitrarily set to 5.

\begin{equation}
stability = (r_M - r_m) \frac{|k_{r_M} - k_{r_m}|}{N} < 2
\label{eq:stability}
\end{equation}

\begin{figure}[!h]
\centering
\includegraphics[width=0.9\columnwidth]{../figures2/stability_lowres.eps}
\caption{stability}
\label{fig:stability}
\end{figure}

\section{Results}
\label{results}
The {\it brain speed reader} is an attempt to test a fast-paced brain-computer interface, which bets that humans can self-regulate their brain wave modulations, in order to control the rate of stimuli in a rapid serial visual presentation setting. Self-regulation achievement is a token of concentration on the coherent sequence of stimuli displayed. Therefore, to evaluate the {\it brain speed reader}, we have primarily focused on the capacity to self-regulate. We tested two opposed designs: {\it bsr+} ($\alpha = 0.005$) and {\it bsr-} ($\alpha = -0.005$). We compared the brain speed reader treatments to the constant RSVP rate [$X(t)  = 125$ ms/word], for perceived comfort and comprehension, keeping in mind that speed reading, and RSVP speed reading in particular, are somewhat unusual: In our 21 participants sample, only one had practiced speed reading before taking the experiment.

\subsection{Self-regulation achievement \& perceived comfort}
With no previous research or preliminary results at hand, we had no {\it a priori} hypothesis on whether users can achieve self-regulation ($stability <  2$) at all, for both treatments  {\it bsr+} and {\it bsr-}, or only for one of them. We found that 72\% of the participants who took {\it bsr+} achieved self-regulation at least once. Similarly, 58\% of the participants could achieve self-regulation at least once in the {\it bsr-} treatment. These results show that capabilities by humans to achieve self-regulation are high even without prior training. Self-regulation with {\it bsr+} treatment appears to be only slightly more prevalent, suggesting that one RSVP rate update method is easier than the other. To confirm this hypothesis, we considered participants who achieved self-regulation for both {\it bsr+} and {\it bsr-}. We found that 47\% of the participants could self-regulate with both opposed treatments. This result suggests in turn that, for roughly half users tested, the brain has enough plasticity to adapt the brain wave modulations, in order to successfully control the RSVP rate. Since there is no {\it ex-ante} training in the experiment, we can assert that this adaptation occurs quickly, reflecting a form of brain dynamical plasticity.


\begin{table}[h!]
	\centering
	\caption{Ordinary least square regression of stability.}
	\begin{tabular}{lcccc} \hline
		& (1) & (2) \\%& (3) & (4) \\
		VARIABLES & -Stability & -Stability \\%& $V_{it}$ & $V_{it}$ \\ \hline
		&  &  \\%&  &  \\
		Age & {\bf -0.0300**} & {\bf -0.0345**} \\ %& -2.310*** & -1.236** \\
		& {\it (0.0073)} & {\it (0.0087)} \\%& (0.603) & (0.515) \\
		Normal Read Rate & {\bf 0.1650*} & {\bf 0.0920*}  \\%& -23.72*** & -7.188** \\
		& {\it (0.0628)} & {\it (0.067)} \\%& (2.152) & (3.473) \\
		Text Length & {\bf -0.0013***} &  {\bf -0.0012***} \\%& -3.312*** & -3.758*** \\
		& {\it (0.0004)} & {\it (0.0005)} \\%& (1.239) & (1.128) \\
		Reading Pleasure & {\bf 0.3034*} & {\bf 0.2436*} \\ %& -0.0312 & -0.0321* \\
		& {\it (0.0591)} & {\it (0.0650)} \\ 
		Speed Reading Comfort & {\bf -0.0726*}  & {\bf -0.1217*} \\%& 0.106** & 0.0755* \\
		& {\it (0.0355)} & {\it (0.0346)}  \\%& (0.0431) & (0.0406) \\
		Familiar Topic & {\bf -0.091*} &  \\%& 16.75*** & -7.414 \\
		& {\it (0.040)} & \\%& (1.339) & (5.698) \\
		Constant & -1.3213 & -0.6579 \\%& 190.3*** & 136.5*** \\
		& {\it (0.5390)} & {\it (0.5565)} \\%& (23.17) & (26.17) \\
		&  &  \\%&  &  \\
		
		R-squared & 0.8148 & 0.8930 \\%& 0.319 & 0.647 \\
		p-value & 0.0042 & 0.0083\\
		Observations & 14 & 14 \\%& 1,212 & 1,212 \\
		%Program FE & No & No \\%& No & Yes \\ \hline
		&  &  \\
		\hline
		\multicolumn{3}{l}{ Robust standard errors in parentheses} \\
		\multicolumn{3}{l}{ *** p$<$0.001, ** p$<$0.01, * p$<$0.05} \\
	\end{tabular}
	\label{tab:reg}
\end{table}

We have considered an arbitrary threshold for stability ($stability < 2$), but stability is a continuous variable from 0 (perfect stability, i.e., constant rate) to $\infty$ [here, $max(stability) = 5$ by convention, see section Methods]. Considering participants who achieved stability ($ 0 < stability < 2$), we take stability as a dependent variable, and using a ordinary least square (OLS) model we have investigated which demographic, text, and reported comfort control variables best influence stability (c.f. Table \ref{tab:reg}). We found that reading speed in normal setting, familiarity with the topic, and reading pleasure are highly associated with increased stability. On the contrary, age and text length negatively influences stability achievement. Surprisingly, speed reading comfort and familiarity with the topic (as reported by participants) is also negatively associated with stability. The negative impact of speed reading comfort may suggest that less stability introduces some additional degree of freedom on how users control the RSVP rate (yet at the expense of reading pleasure). We have no explanation for the negative impact of topic familiarity. We have tested a second model (2) by removing the familiarity with the topic. All other parameters remain almost unchanged.

\subsection{Reading speed}
Reading speed is another important factor. During the preliminary tasks, participants were asked to read an article excerpt entirely displayed on the screen. We find that it took $240\pm50$ milliseconds/word to read this piece of text, even though there were no follow-up comprehension questions, unlike for the RSVP treatments for which participants were informed of a detailed description of follow-up comprehension questions. For those who achieved stability during {\it bsr+} and {\it bsr-} treatments, the average RSVP rate was found to be very close to 125 milliseconds per word with 5th and 95th percentiles, respectively 110  and 142 ms/word. The confidence interval of average RSVP rate found here is similar but smaller than what was found as the comfort zone in previous research (80 to 200 ms/word) \cite{kujala2007phase}. Another explanation would be that the proximity of the average RSVP rate with the initial rate $X(t=0) = 125$, may explain why the average rate was found to be very close to this value. 

\subsection{Comprehension}
Now we turn to reading comprehension as tested from (i) short summary, (ii) free recall of proper nouns, and (iii) selecting common nouns. We find limited support suggesting that more stability increases the capacity to produce a meaningful summary (Spearman correlation $\rho= 0.35$, $p = 0.069$). However, we found no effect on proper and common nouns recall.



%\input{../sections/data}
%\input{../sections/dynamicsAnalysis}
%\input{../sections/tailAnalysis}
%\section{Discussion}
\label{discussion}


\subsubsection{Summary of results}

\subsubsection{Reconnect with previous results obtained in cognitive neuro-sciences}

\subsubsection{Neurofeedback}
30. Heinrich, H., Gevensleben, H. \& Strehl, U. Annotation: neurofeedback - train
your brain to train behaviour. J. Child Psychol. Psychiatry 48, 3�16 (2007).

\textcolor{red}{\bf [maybe this should be left for a subsequent iteration involving subjects repeating the experiment.]}


Comment by a participant (Tap): ``It would be great to have a way to track eyes, so that when the user attention leaves the screen, the brain speed reader stops"




\subsubsection{Other media}
- audio

- cartoons

- etc.

\subsection{Curing Diseases}
- visual impairment

- adhd



% Results and Discussion can be combined.
% You may title this section "Methods" or "Models". 
% "Models" is not a valid title for PLoS ONE authors. However, PLoS ONE
% authors may use "Analysis" 
%\section*{Materials and Methods}

% Do NOT remove this, even if you are not including acknowledgments
%\section*{Acknowledgments}

%\section*{References}
% The bibtex filename



\bibliography{../bib/bsr,../bib/decoding,../bib/tmaillart,../bib/neurofeedback}
\section{Figures}

\begin{figure}[H]
\centering
\includegraphics[width=7cm]{../figures2/apparatus.eps}
\caption{Brain speed-reader apparatus: {\bf (a)} Words are displayed and read one after the other at a given rate. {\bf (b)} the EEG signal is recorded through a consumer grade device (here the {\it Neurosky Mindwave}). {\bf (c)} The EEG signal is turned every 0.5 seconds into a power spectrum through a Fourier transform, {\bf (d)} the characteristics of the power spectrum are compressed into a single value characteristic entropy $s$ value. {\bf (e)} A new rate of word display is updated by taking into its current value and $s$. {\bf (f)} The rate of word display is updated accordingly.}
\label{fig:apparatus}
\end{figure}

\begin{figure}[H]
\centering
\includegraphics[width=17cm]{../figures2/balance.eps}
\caption{{\bf a.} Schematic representation of the balance of rate change $\Delta_{rate}$ as a function of word length $l_{words}$. The color gradient shows schematically the word density conditioned on their length. This is the canonical or most desirable situation: Around the mean word length the deterministic component of $\Delta_{rate}$ is close to zero. For words with length smaller, the deterministic part of the rate increases, while for words with length larger than the mean, the rate is decreased. The dotted line shows another possible configuration, with acceleration occurring when words are longer. Empirical evidence for the latter case is shown  in Figure \ref{fig:examples} for some typical successful and failed attempts to maintain a balance. ({\bf c}) The joint probability density function $pdf( \Delta_{rate} \times l_{words})$ is well balanced, yet skewed, showing that good control is achieved, along with a good capacity to change the rate of word display.}
\label{fig:apparatus}
\end{figure}

%The linear regression of the average $\Delta_{rates}$ for each word length (for $l_{words} < 10$) exhibits a slope $= 4(1)\times10^{-3}$ ($p < 0.01$). The intersection of $l_{words}(\Delta_{rates} =0) = 4.43$ very close to the average word length (in the text). The error bars show the dispersion (standard deviation of  $\Delta_{rates}$ for each word length. This dispersion is rather large reflecting the stochastic nature of complex brain activation sand the coarse measure obtained from the single electrode EEG headset.

\begin{figure}[H]
\centering
\includegraphics[width=12cm]{../figures2/examples.eps}
\caption{Four examples of successful and failed neuro-feedback control strategies. For each case, three panels are shown (from left to right): (i) Evolution of rate at each displayed word, (ii) rate change as a function of word length at each time step, and (iii)  rate change in the vicinity of large words (9 or more characters, red line), versus words with smaller than 5 characters (blue). The 90\% confidence intervals (light blue area) are obtained by replacement bootstrapping (100 samples of same size as large words are randomly drawn from small words). {\bf a.} Illustration of a very well controlled RSVP, with a sharp and localized drop of word presentation rate around the time large word occurrence. {\bf b.} Opposite strategy with rate increased around large words. {\bf c.} Yet another neuro-feedback strategy with the rate being controlled after the word has occurred. {\bf d.} Failed strategy: Compared to {\bf a}, {\bf b} and {\bf c} the rate change is consistently negative, hence dragging RSVP towards the lower rate limit. Note also in {\bf c} how the rate change as a function of word length (middle panel) is unbalanced around the 0-rate change (horizontal black line) and the mean word length (vertical black line), on the  contrary to {\bf a}, {\bf b} and {\bf c}.}
\label{fig:examples}
\end{figure}


\begin{figure}[H]
\centering
%\includegraphics[width=12cm]{../figures2/examples.eps}
\caption{Here a figure on how the rate is influenced by the power septrum. The idea is to cherry pick moments of high rate change, and look how the power spectrum (and entropy) influences these changes (keep in mind the smoothing, which should reduce the effects of pSpectrum changes on the rate.).}
\label{fig:S_vs_rate}
\end{figure}

\begin{figure}[H]
\centering
%\includegraphics[width=12cm]{../figures2/examples.eps}
\caption{To measure whether there is an effect in the constant rate case, one must first reverse engineer how the rate influences some power spectrum, and how it influences the rate}
\label{fig:constant_rate}
\end{figure}





%\section{Figures}

\begin{figure}[H]
\centering
\includegraphics[width=7cm]{../figures2/apparatus.eps}
\caption{Brain speed-reader apparatus: {\bf (a)} Words are displayed and read one after the other at a given rate. {\bf (b)} the EEG signal is recorded through a consumer grade device (here the {\it Neurosky Mindwave}). {\bf (c)} The EEG signal is turned every 0.5 seconds into a power spectrum through a Fourier transform, {\bf (d)} the characteristics of the power spectrum are compressed into a single value characteristic entropy $s$ value. {\bf (e)} A new rate of word display is updated by taking into its current value and $s$. {\bf (f)} The rate of word display is updated accordingly.}
\label{fig:apparatus}
\end{figure}

\begin{figure}[H]
\centering
\includegraphics[width=17cm]{../figures2/balance.eps}
\caption{{\bf a.} Schematic representation of the balance of rate change $\Delta_{rate}$ as a function of word length $l_{words}$. The color gradient shows schematically the word density conditioned on their length. This is the canonical or most desirable situation: Around the mean word length the deterministic component of $\Delta_{rate}$ is close to zero. For words with length smaller, the deterministic part of the rate increases, while for words with length larger than the mean, the rate is decreased. The dotted line shows another possible configuration, with acceleration occurring when words are longer. Empirical evidence for the latter case is shown  in Figure \ref{fig:examples} for some typical successful and failed attempts to maintain a balance. ({\bf c}) The joint probability density function $pdf( \Delta_{rate} \times l_{words})$ is well balanced, yet skewed, showing that good control is achieved, along with a good capacity to change the rate of word display.}
\label{fig:apparatus}
\end{figure}

%The linear regression of the average $\Delta_{rates}$ for each word length (for $l_{words} < 10$) exhibits a slope $= 4(1)\times10^{-3}$ ($p < 0.01$). The intersection of $l_{words}(\Delta_{rates} =0) = 4.43$ very close to the average word length (in the text). The error bars show the dispersion (standard deviation of  $\Delta_{rates}$ for each word length. This dispersion is rather large reflecting the stochastic nature of complex brain activation sand the coarse measure obtained from the single electrode EEG headset.

\begin{figure}[H]
\centering
\includegraphics[width=12cm]{../figures2/examples.eps}
\caption{Four examples of successful and failed neuro-feedback control strategies. For each case, three panels are shown (from left to right): (i) Evolution of rate at each displayed word, (ii) rate change as a function of word length at each time step, and (iii)  rate change in the vicinity of large words (9 or more characters, red line), versus words with smaller than 5 characters (blue). The 90\% confidence intervals (light blue area) are obtained by replacement bootstrapping (100 samples of same size as large words are randomly drawn from small words). {\bf a.} Illustration of a very well controlled RSVP, with a sharp and localized drop of word presentation rate around the time large word occurrence. {\bf b.} Opposite strategy with rate increased around large words. {\bf c.} Yet another neuro-feedback strategy with the rate being controlled after the word has occurred. {\bf d.} Failed strategy: Compared to {\bf a}, {\bf b} and {\bf c} the rate change is consistently negative, hence dragging RSVP towards the lower rate limit. Note also in {\bf c} how the rate change as a function of word length (middle panel) is unbalanced around the 0-rate change (horizontal black line) and the mean word length (vertical black line), on the  contrary to {\bf a}, {\bf b} and {\bf c}.}
\label{fig:examples}
\end{figure}


\begin{figure}[H]
\centering
%\includegraphics[width=12cm]{../figures2/examples.eps}
\caption{Here a figure on how the rate is influenced by the power septrum. The idea is to cherry pick moments of high rate change, and look how the power spectrum (and entropy) influences these changes (keep in mind the smoothing, which should reduce the effects of pSpectrum changes on the rate.).}
\label{fig:S_vs_rate}
\end{figure}

\begin{figure}[H]
\centering
%\includegraphics[width=12cm]{../figures2/examples.eps}
\caption{To measure whether there is an effect in the constant rate case, one must first reverse engineer how the rate influences some power spectrum, and how it influences the rate}
\label{fig:constant_rate}
\end{figure}





%\setcounter{section}{0}
%\renewcommand\thesection{\Alph{section}}
%\renewcommand\thesubsection{\thesection.\arabic{subsection}}
%\renewcommand\thesubsubsection{\alph{subsubsection}}
%\clearpage
%\begin{center}
%{\bf \Huge Supplementary Materials}
%\end{center}
%\vspace{2cm}


%\section*{Tables}
%\begin{table}[!ht]
%\caption{
%\bf{Table title}}
%\begin{tabular}{|c|c|c|}
%table information
%\end{tabular}
%\begin{flushleft}Table caption
%\end{flushleft}
%\label{tab:label}
% \end{table}

\end{document}

