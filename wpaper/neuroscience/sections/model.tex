\section{Model}
To achieve brain self-regulation under constraint of varying word difficulty (incl. visual processing effort and semantic comprehension), participants locally modify the amplitude of frequencies across the full spectrum of their brain waves.  This modification of the frequency spectrum is captured by an entropy measure (c.f., method), which in turn triggers a word presentation rate change.\\

\paragraph{\bf steady-state visual evoked potentials (SSVEP)}

Since there is a direct correspondence between the rate, and the local entropy, which in turn is a measure of variation of the frequency spectrum, in order to understand how participants achieve brain self-regulation, it is therefore sufficient to reverse engineer the causality chain: (i) a specific word appeared (in context), (ii) trigger a rate change, (iii) itself induced by an entropy change, which is measure of (iv) temporal local brain activity change, which change is triggered by some cognitive processing, which involves some visual processing and semantic analysis, involving working memory etc. {\bf [some work needed here, but stay brief]}.\\

Given that participants can stabilize the rate of word presentation, yet with high fluctuations, we shall consider what in the sequence of stimuli may induce rate change rate. We hypothesize that {\bf word length is the most important factor determining rate change}. Among many other possible factors explaining why a long word may trigger a more important change of amplitude, we consider that word rarity is overall an inverse function of word length, and word length requires more visual processing.\\

{\bf Hypothesis 1:} So our working hypothesis is that the occurrence of word deviating from the average word size is more likely to trigger a significant change of the brain wave frequency distribution, hence of the entropy, hence of the rate of word presentation.\\

{\bf Hypothesis 2a:} Given that a participant achieves stationary word presentation, we can find a recurrent pattern change in the power spectrum, and thus in the waveform around the focal word triggering change.\\

{\bf Hypothesis 2b:} On the contrary, participants would don't achieve stationary word presentation, don't exhibit the same regularity. We also test what happens in the first 100 words of the text, when stationarity has not been achieved yet.\\

{\bf (highly challenging, maybe to be left for discussion) Hypothesis 3:} The regularity shall reflect the compounded effects of the various stages of information processing in the brain, including precursors (before the focal word has appeared), which we cannot disentangle here.\\



\paragraph{Reading word triggers some ERP (N400?)} Reading is known to trigger some waveform

%\paragraph{Reading and RSVP}



